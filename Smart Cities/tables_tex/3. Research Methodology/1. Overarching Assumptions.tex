% Please add the following required packages to your document preamble:
% \usepackage{graphicx}
\begin{table}[tbh!]
\caption{Overarching Assumptions and Limitations for All Models}
\label{tab: overarching assumptions}
\resizebox{\columnwidth}{!}{%
\begin{tabular}{|p{0.5cm}|p{8cm}|p{10cm}|p{0.6cm}|}
\hline
\textbf{No.} & \textbf{Assumption}                                                                                                                                                                                                                  & \textbf{Limitation}                                                                                                                                                                  & \textbf{Ref.} \\ \hline
\textbf{1}   & The fluid acts as a Newtonian fluid and is incompressible eliminating some of the more complex fluid dynamics. This is because the fluid does not have the ability to deform or change its shape easily in response to an external force. & The fluid won't be able to transmit shear waves, it cannot exhibit turbulence and it may not fully represent the HD-Fluid R-19.                                                     &     \cite{nonnewtonianfluid}          \\ \hline
\textbf{2}   & The penstock is inelastic and is made of a relatively rigid material. Simplifying the analysis within it.                                                                                                                           & This may neglect some fluid dynamic effects caused by changes in the penstock under high fluid pressure or flow rates.                                                              &  \cite{mainmodelpaper}     \\ \hline
\textbf{3}   & The penstock area is constant throughout the flow, and any bends or sudden contractions don't directly effect the fluid properties.                                                                                                  & This will ignore any dynamic fluid effects that could cause flow to behave differently and potentially lose more pressure head.                                                     &               \\ \hline
\textbf{4}   & The PHES systems start-up or shut-down is over a very short period compared to its normal running conditions so the transient effects will be assumed to be nil.                                                                                                                        & If there are large losses or effects on the fluid when the system is started or shut down they will be ignored, this includes accelerations of the fluid.                                &               \\ \hline
\textbf{5}   & The fluid flow is in steady state, so there is no acceleration of the fluid at any point. It also means the fluids velocity is the same at all points withing the cross-sectional area.                                              & This becomes a major limitation if the systems start-up effects need to be modelled, as well as if the penstock changes shape at any point.                                       &           \cite{improvedmomentummodel}    \\ \hline
\textbf{6}   & There is no pressure gradient in the same cross-sectional plane of flow. This is a valid assumption as the losses due to turbulence are usually very small and can be ignored.                                                                                                                                                            & This is only true when flow is steady, so any unsteady flow or turbulence won't be accounted for. Additionally, any pressure gradients formed by bends or contractions are ignored. This becomes a limitation as losses to turbulence are ignored. &  \cite{mainmodelpaper}     \\ \hline
\textbf{7}   & The fluid is incompressible and its density remains constant throughout its flow.                                                                                                                                                    & Given the fluid flowing isn't a gas this is a fair assumption and doesn't limit the model in any way.                                                                               & \cite{fluidmechanicstextbook}     \\ \hline
\textbf{8}   & The fluid flow is laminar meaning it is regular and smooth with no turbulence.                                                                                                                                                       & The effects of turbulence aren't taken into account, however, this common limitation won't have huge effects on the result of flow speeds.                                          & \cite{improvedmomentummodel}    \\ \hline
\end{tabular}%
}
\end{table}