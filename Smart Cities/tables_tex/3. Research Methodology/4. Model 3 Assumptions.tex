\begin{table}[tbh!]
\caption{Model 3 Additional Assumptions and Limitation}
\label{tab: model 3 assumption}
\resizebox{\columnwidth}{!}{%
\begin{tabular}{|p{0.5cm}|p{8cm}|p{10cm}|p{0.6cm}|} 
\hline
\textbf{No.} & \textbf{Assumption}                                                        & \textbf{Limitation}                                                                                                                                           & \textbf{Ref.} \\ \hline
\textbf{16}   & Assume the load torque is known and doesn't have any influence on the power generated by the turbine itself.                             &    In the real world there will be losses corresponding to this load torque. If they are ignored the power output may be slightly higher than real data.  &        \cite{mainmodelpaper}       \\ \hline
\textbf{17}   & The stationary vanes and guide vanes are assumed to act as loss factors to show their effect on the total mass flow rate. The guide vanes value will change with respect to its angle. & This may cause a false estimate of the losses and will definitely need to be fine tuned with experimental data. &   \cite{turbinemodel}  \\ \hline
\textbf{18}   & The loss due to the guide vane angles is assumed to follow Eq.~(\ref{eqn: guide vane loss}), this is because it is known that closing the guide vane angles will reduce the flow rate \cite{fluidmechanicstextbook}. & Without experimental testing this loss coefficient will not accurately represent how much the flow rate actually decreases or increases, however, it will show the trend that is expected.  & \cite{fluidmechanicstextbook} \\ \hline
\end{tabular}%
}
\end{table}